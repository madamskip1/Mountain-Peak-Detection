\cleardoublepage % Zaczynamy od nieparzystej strony
\streszczenie

Celem niniejszej pracy dyplomowej magisterskiej było zbadanie oraz przetestowanie algorytmów cyfrowego przetwarzania obrazów do identyfikacji szczytów górskich na obrazie w czasie rzeczywistym. Prace badawcze były realizowane na stworzonej na potrzeby projektu platformie testowej na urządzeniach mobilnych z systemem Android. Opracowany proces rozpoznawania gór może  zostać wykorzystany również w innych konfiguracjach sprzętowych.

Opierając się na analizie literatury oraz podobnych, istniejących już rozwiązaniach przygotowany został wstępny proces rozpoznawania szczytów, który w dalszych iteracjach prac badawczych był modyfikowany dzięki poszerzanej wiedzy, a także na podstawie wyników testów oraz porównań algorytmów. Została przygotowana na potrzeby pracy dyplomowej aplikacja, która w momencie wykonywania zdjęcia zapisuje dodatkowe dane geolokalizacyjne na jego temat. Dzięki temu stworzono zbiór statycznych zdjęć gór z~odpowiednimi danymi. Zasilana nimi platforma testowa pozwalała sprawdzić działanie przygotowanych metod, a także wybrać optymalne w kontekście procesu rozpoznawania szczytów górskich. Testy były zorientowane na czas wykonania algorytmów związany z~działaniem projektu w czasie rzeczywistym, ale również na ich jakość i skuteczność.

Proces rozpoznawania szczytów górskich zaproponowany w ramach pracy dyplomowej składa się z kilku elementów. Na początku zbierane są dane z sensorów urządzenia. Na ich podstawie generowany jest trójwymiarowy model płaszczyzny Ziemi z perspektywy obserwatora. Wykorzystywane do tego są \textit{numeryczne modele terenu} oraz interfejs \textit{OpenGL}. Na tak wygenerowanej panoramie określane są widoczne góry. W tym celu przeprowadzane jest filtrowanie danych dotyczących szczytów górskich przy pomocy metod usuwania powierzchni niewidocznych - \textit{frustum culling} oraz \textit{occlusion culling}. Wykorzystując algorytm detekcji krawędzi \textit{Canny} tworzone są na podstawie panoramy oraz zdjęcia wejściowego obrazy binarne zawierające kontury gór. Przy użyciu metody \textit{dopasowania obrazów na podstawie szablonu} porównywane są odpowiednie fragmenty zdjęć potencjalnie zawierające dane szczyty. Na podstawie uzyskanych wyników stwierdzana jest widoczność poszczególnych szczytów górskich na kolejnych klatkach nagrania. 

Opracowany proces został poddany weryfikacji końcowej w warunkach rzeczywistych. Potwierdziła ona zdolność takiego cyklu do prawidłowej identyfikacji szczytów górskich oraz wykazała możliwość jej działania w czasie rzeczywistym.




\slowakluczowe rozpoznawanie szczytów górskich, rejestracja obrazu, wyrównanie zdjęć, dopasowanie obrazu na podstawie szablonu, numeryczne modele terenu, grafika trójwymiarowa, przetwarzanie cyfrowe obrazu, czas rzeczywisty