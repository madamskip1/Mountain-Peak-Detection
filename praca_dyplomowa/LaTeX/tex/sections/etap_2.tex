\newpage

\section{Drugi etap implementacyjny} \label{sec:drugi_etap}

Drugi etap implementacji projektu polegał na wykorzystaniu wiedzy zdobytej w trakcie poprzedniej części, celem stworzenia finalnego oprogramowania na urządzenia mobilne. Takimi informacjami były możliwość wyboru algorytmów, wyniki ich testów, rozwiązane problemy czy potrzebne dane. Implementacja na tym etapie była już całkowicie oparta na~języku Java. 

W tym przypadku implementowane były już wyłącznie wybrane podczas testów, najbardziej optymalne rozwiązania. Jednak celem otwartości na zmiany oraz umożliwienia wykorzystania i testowania innych metod, architektura oprogramowania została zmodularyzowana. Dodatkowo, poszczególne kroki całego procesu zaimplementowane zostały z~wykorzystaniem abstrakcji oraz interfejsów. Dzięki temu możliwe jest dodanie kolejnych algorytmów bez potrzeby ingerowania w trzon projektu i procesu rozpoznawania szczytów. Zostały również stworzone dodatkowe aktywności i widoki umożliwiające sprawdzenie poszczególnych funkcjonalności takich jak renderowanie modelu w danym punkcie, testowanie detekcji krawędzi czy kontrola poprawności działania sensorów urządzenia. 



\subsection{Podsumowanie etapu}

Wynikiem drugiego etapu była działająca aplikacja umożliwiająca rozpoznawanie szczytów górskich na obrazie w czasie rzeczywistym. Testy pokazały, że spełnia ona założoną funkcjonalność oraz wymagane działanie w określonym czasie. Dodatkowo udało się wykryć i naprawić efekt migotania etykiet na nagraniach, dzięki czemu rozwiązanie stało się bardziej czytelne i przyjazne użytkownikowi. 