\begin{table}[!h]
  \centering
  \small
  \caption{Wpływ operacji morfologicznej dylacji na algorytm dopasowania na podstawie szablonu.}
\resizebox{\textwidth}{!}{%
  \begin{tabularx}{\linewidth}{|l|*{4}{>{\centering\arraybackslash}X|}}
    \hline
    &
     \hspace{0em}\textbf{Dokładność}  & 
      \textbf{Precyzja}   & 
      \textbf{Czułość} &
      \textbf{F1} \\
    \hline
    \hline
    
    {\begin{tabular}[c]{@{}l@{}}Dylacja na\\ renderze\end{tabular}}  & {\begin{tabular}[c]{@{}c@{}}$29/32$\\ $(90,6\%)$\end{tabular}} & {\begin{tabular}[c]{@{}c@{}}$26/32$\\ $(81,3\%)$\end{tabular}} & {\begin{tabular}[c]{@{}c@{}}$20/32$\\ $(62,5\%)$\end{tabular}} & {\begin{tabular}[c]{@{}c@{}}$26/32$\\ $(81,3\%)$\end{tabular}} \\
    \hline
    {\begin{tabular}[c]{@{}l@{}}Dylacja na\\ zdjęciu rzeczywistym\end{tabular}} &  {\begin{tabular}[c]{@{}c@{}}$26/32$\\ $(75,0\%)$\end{tabular}} & {\begin{tabular}[c]{@{}c@{}}$24/32$\\ $(62,5\%)$\end{tabular}} & {\begin{tabular}[c]{@{}c@{}}$20/32$\\ $(85,7\%)$\end{tabular}} &  {\begin{tabular}[c]{@{}c@{}}$22/32$\\ $(68,8\%)$\end{tabular}} \\
    \hline
    {\begin{tabular}[c]{@{}l@{}}Dylacja na\\ obu obrazach\end{tabular}} &  {\begin{tabular}[c]{@{}c@{}}$28/32$\\ $(87,5\%)$\end{tabular}} & {\begin{tabular}[c]{@{}c@{}}$29/32$\\ $(90,6\%)$\end{tabular}} & {\begin{tabular}[c]{@{}c@{}}$21/32$\\ $(65,6\%)$\end{tabular}} &  {\begin{tabular}[c]{@{}c@{}}$25/32$\\ $(78,1\%)$\end{tabular}} \\
    \hline

    
  \end{tabularx}%
}
  \label{tab:dilation-comparision-template}
\end{table}

